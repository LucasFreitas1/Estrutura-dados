\documentclass[12pt]{article}
\usepackage[T1]{fontenc}
\usepackage[linesnumbered,portuguese]{algorithm2e} %ruled
\usepackage{amsmath} % habilita fun��es matem�ticas do AMS-LaTeX
%\usepackage[brazil]{babel} % dicion�rio para o ".tex", tradu��es na compila��o
\usepackage[brazilian]{babel}
\usepackage{xcolor}
\usepackage{graphicx}

\usepackage{exercise}
\usepackage[latin1]{inputenc} % acentua��o usando UTF8 (ou senao, use latin1)

\setlength{\topmargin}{-0.75in}
\setlength{\textheight}{9in}
\setlength{\textwidth}{6.5in}
\setlength{\oddsidemargin}{0in}
\setlength{\evensidemargin}{0in}



\begin{document}

\begin{table}

  \begin{tabular}{|p{11cm}|p{5cm}|}
  \hline \textbf{Nome}: & \textbf{Matr�cula}:\\
  \hline \textbf{Disciplina}: Estrutura de Dados & \textbf{Data}: 27/02/2012\\
  \hline \textbf{Professor}:  Alessandro & \textbf{Turma}: TAD / TRC 4AN\\
  \hline
  \end{tabular}
  
\end{table}

  \large \def\ExerciseName{Lista de Exerc�cios 0}

\begin{Exercise*}

\Question Escreva um programa que leia os elementos de uma matriz $10 \ x \ 10$ e apresente separadamente:
\begin{enumerate}
	\item Todos os elementos, exceto os da diagonal principal.
    \item Os elementos acima da diagonal principal.
\end{enumerate}

\Question Escreva um programa que dado uma matriz $M_{4 x 7}$ apresente a transposta da matriz M. A matriz transposta � gerada trocando linha por coluna.


\Question Escreva um algoritmo que leia e armazene os elementos de uma matriz inteira $M_{10 x 10}$ e apresente os seus elementos segundo as seguintes regras:

\begin{enumerate}
   \item troque a segunda linha pela oitava linha;
   \item troque a quarta linha pela d�cima linha;
   \item troque a diagonal principal pela diagonal secund�ria
\end{enumerate}

\Question Escreva um programa que armazene valores inteiros para uma matriz de ordem quatro. Apresente toda a matriz e o determinante da matriz. O programa deve impedir a entrada do valor zero para qualquer elemento que se encontre na diagonal secund�ria ou abaixo da diagonal secund�ria e s� deve permitir a entrada do zero para todo elemento que se encontre acima da diagonal secund�ria.

\textbf{Considera��es}: A solu��o desse algoritmo se torna simples desde que voc� saiba que, quando uma matriz tem um tri�ngulo de zeros, em rela��o a Diagonal Secund�ria (DS), o c�lculo do determinante se faz da seguinte forma:

\begin{enumerate}
   \item Multiplicam-se os elementos que est�o na DS;
   \item Se a matriz for de ordem par, ser� igual ao resultado do produto;
   \item Se a matriz for de ordem �mpar, ser� igual ao resultado do produto negativado.
\end{enumerate}


\Question Escreva um algoritmo que leia os valores de uma matriz de ordem cinco e verifique se ela � ou n�o uma matriz identidade.
Matriz identidade � uma matriz onde todos os elementos da diagonal principal (DP) s�o iguais a 1 e os outros iguais a zero.

\end{Exercise*}

\end{document}