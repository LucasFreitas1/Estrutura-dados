\documentclass[12pt]{article}
\usepackage[T1]{fontenc}
\usepackage[linesnumbered,portugues]{algorithm2e} %ruled
\usepackage{amsmath} % habilita fun��es matem�ticas do AMS-LaTeX
%\usepackage[brazil]{babel} % dicion�rio para o ".tex", tradu��es na compila��o
\usepackage[brazilian]{babel}
\usepackage{xcolor}
\usepackage{graphicx}

\usepackage{exercise}
\usepackage[latin1]{inputenc} % acentua��o usando UTF8 (ou senao, use latin1)

\setlength{\topmargin}{-0.75in}
\setlength{\textheight}{9in}
\setlength{\textwidth}{6.5in}
\setlength{\oddsidemargin}{0in}
\setlength{\evensidemargin}{0in}



\begin{document}

\pdfinfo{
   /Author (Alessandro Leite)
   /Title  (Lista de Exerc�cios)
   /CreationDate (\today)
   /Subject (Lista de Exerc�cios 3)
   /Keywords (Exerc�cios, Estrutura de Dados)
}


\large \def\ExerciseName{Lista de Exerc�cios 3}

\begin{Exercise*}

\Question Construa uma fun��o que retorne o tamanho de uma lista est�tica.

\Question Dada uma lista est�tica de tamanho $m$, construa uma fun��o que inverta a lista.

\Question Dada uma lista est�tica de tamanho $m$, construir uma fun��o que retorne uma \textit{sublista} de tamanho $n$ a partir da posi��o $p$. Compute a complexidade de tempo da sua fun��o.

\Question Dada duas listas est�ticas de tamanho $m_1$ e $m_2$, respectivamente, construir uma fun��o para intercalar as duas listas, gerando uma terceira.

\Question Dada duas listas est�ticas de tamanho $m_1$ e $m_2$, respectivamente, construir uma fun��o para concatenar as duas listas, gerando uma terceira.

\Question Uma maneira usual de representar conjuntos � listando seus elementos. Implemente uma aplica��o que ofere�a as opera��es usuais de conjuntos (uni�o, intersec��o e diferen�a), considerando que cada um dos conjuntos � representado por uma lista linear.
		
\end{Exercise*}

\end{document}