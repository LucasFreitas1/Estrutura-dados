\documentclass[12pt]{article}
\usepackage[T1]{fontenc}
\usepackage[linesnumbered]{algorithm2e} %ruled
\usepackage{amsmath} % habilita fun��es matem�ticas do AMS-LaTeX
%\usepackage[brazil]{babel} % dicion�rio para o ".tex", tradu��es na compila��o
\usepackage[brazilian]{babel}
\usepackage{xcolor}
\usepackage{graphicx}

\usepackage{listings}

\usepackage{exercise}
\usepackage[latin1]{inputenc} % acentua��o usando UTF8 (ou senao, use latin1)

\setlength{\topmargin}{-0.75in}
\setlength{\textheight}{9in}
\setlength{\textwidth}{6.5in}
\setlength{\oddsidemargin}{0in}
\setlength{\evensidemargin}{0in}

\definecolor{Brown}{cmyk}{0,0.81,1,0.60}
\definecolor{OliveGreen}{cmyk}{0.64,0,0.95,0.40}
\definecolor{CadetBlue}{cmyk}{0.62,0.57,0.23,0}

\lstset{language=Java,basicstyle=\footnotesize,
  keywordstyle=\ttfamily\color{brown},%OliveGreen
 identifierstyle=\ttfamily\color{CadetBlue}\bfseries, 
 commentstyle=\color{Brown},
 stringstyle=\ttfamily,
 showstringspaces=false}



\begin{document}

\pdfinfo{
   /Author (Alessandro Leite)
   /Title  (Estrutura de Dados)
   /CreationDate (\today)
   /Subject (Lista de Exerc�cios 6)
   /Keywords (Exerc�cios, Estrutura de dados)
}

\large \def\ExerciseName{Lista de Exerc�cios 7}

\begin{Exercise*}

\Question Escreva uma fun��o recursiva que receba dois n�meros inteiros positivos \textit{k} e \textit{n} e calcule o valor de $k^n$. 

\Question A seguinte fun��o calcula o M�ximo Divisor Comum (MDC) dos inteiros positivos \textit{m} e \textit{n}. Escreva uma fun��o recursiva equivalente.

\begin{lstlisting}
int mdc(int m, int n){
 int r;
 do{
   r = m % n;
   m = n; n = r;
 }while(r != 0); 
 return m;
}
\end{lstlisting}

\Question Escreva uma fun��o que converta um n�mero inteiro de decimal para bin�rio.

\Question Usando recurs�o, calcule a soma de todos os valores de um \textit{array}.


\end{Exercise*}

\end{document}