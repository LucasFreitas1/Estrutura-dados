\documentclass[12pt]{article}
\usepackage[T1]{fontenc}
\usepackage[linesnumbered]{algorithm2e} %ruled
\usepackage{amsmath} % habilita fun��es matem�ticas do AMS-LaTeX
%\usepackage[brazil]{babel} % dicion�rio para o ".tex", tradu��es na compila��o
\usepackage[brazilian]{babel}
\usepackage{xcolor}
\usepackage{graphicx}

\usepackage{listings}

\usepackage{exercise}
\usepackage[latin1]{inputenc} % acentua��o usando UTF8 (ou senao, use latin1)

\setlength{\topmargin}{-0.75in}
\setlength{\textheight}{9in}
\setlength{\textwidth}{6.5in}
\setlength{\oddsidemargin}{0in}
\setlength{\evensidemargin}{0in}

\definecolor{Brown}{cmyk}{0,0.81,1,0.60}
\definecolor{OliveGreen}{cmyk}{0.64,0,0.95,0.40}
\definecolor{CadetBlue}{cmyk}{0.62,0.57,0.23,0}

%\lstset{language=C,basicstyle=\footnotesize,
%  keywordstyle=\ttfamily\color{brown},%OliveGreen
% identifierstyle=\ttfamily\color{CadetBlue}\bfseries, 
% commentstyle=\color{Brown},
% stringstyle=\ttfamily,
% showstringspaces=false}

\usepackage{color}
 
\definecolor{dkgreen}{rgb}{0,0.6,0}
\definecolor{gray}{rgb}{0.5,0.5,0.5}
\definecolor{mauve}{rgb}{0.58,0,0.82}
 
\lstset{ %
  language=Java,                % the language of the code
  basicstyle=\footnotesize,           % the size of the fonts that are used for the code
  numbers=left,                   % where to put the line-numbers
  numberstyle=\tiny\color{gray},  % the style that is used for the line-numbers
  stepnumber=2,                   % the step between two line-numbers. If it's 1, each line 
                                  % will be numbered
  numbersep=5pt,                  % how far the line-numbers are from the code
  backgroundcolor=\color{white},      % choose the background color. You must add \usepackage{color}
  showspaces=false,               % show spaces adding particular underscores
  showstringspaces=false,         % underline spaces within strings
  showtabs=false,                 % show tabs within strings adding particular underscores
  frame=single,                   % adds a frame around the code
  rulecolor=\color{black},        % if not set, the frame-color may be changed on line-breaks within not-black text (e.g. commens (green here))
  tabsize=2,                      % sets default tabsize to 2 spaces
  captionpos=b,                   % sets the caption-position to bottom
  breaklines=true,                % sets automatic line breaking
  breakatwhitespace=false,        % sets if automatic breaks should only happen at whitespace
  title=\lstname,                   % show the filename of files included with \lstinputlisting;
                                  % also try caption instead of title
  keywordstyle=\color{blue},          % keyword style
  commentstyle=\color{dkgreen},       % comment style
  stringstyle=\color{mauve},         % string literal style
  escapeinside={\%*}{*)},            % if you want to add a comment within your code
  morekeywords={*,...}               % if you want to add more keywords to the set
}



\begin{document}

\pdfinfo{
   /Author (Alessandro Leite)
   /Title  (Estrutura de Dados)
   /CreationDate (\today)
   /Subject (Lista de Exerc�cios 7)
   /Keywords (Exerc�cios, Estrutura de dados)
}

\large \def\ExerciseName{Gabarito da Lista de Exerc�cios 7}

\begin{Exercise*}

\Question Escreva uma fun��o recursiva que receba dois n�meros inteiros positivos \textit{k} e \textit{n} e calcule o valor de $k^n$. 

\parbox[c]{15.3cm}
{
  \vspace{0.5cm}
  \textbf{Solu��o}\\
}
  
  
\begin{lstlisting}
int potencia(int k, int n)
{
   if (n == 0)
   {
     return 1;
   }else
   {
     n = n - 1;
     return k * potencia(k, n);  
    } 
}
\end{lstlisting}

\Question A seguinte fun��o calcula o M�ximo Divisor Comum (MDC) dos inteiros positivos \textit{m} e \textit{n}. Escreva uma fun��o recursiva equivalente.

\begin{lstlisting}
int mdc(int m, int n){
 int r;
 do{
   r = m % n;
   m = n; n = r;
 }while(r != 0); 
 return m;
}
\end{lstlisting}

\Question Escreva uma fun��o que converta um n�mero inteiro de decimal para bin�rio.

\parbox[c]{15.3cm}
{
  \vspace{0.5cm}
  \textbf{Solu��o}\\
}
\begin{lstlisting}
/**
 * Apresenta a representa��o bin�ria do n�mero n em decimal.*/
void decimal2Binario(int n){
  if (n == 0 || n == 1)
       print(n);
  else{
    decimal2Binario(n/2);
    print(n % 2);        
  }  
}
\end{lstlisting}

\Question Usando recurs�o, calcule a soma de todos os valores de um \textit{array}.

\parbox[c]{15.3cm}
{
  \vspace{0.5cm}
  \textbf{Solu��o}\\
}
\begin{lstlisting}
/**
 * Realiza a soma de todos os elementos do vetor v de tamanho n.
 */
int soma(int v[], int n)
{
  if (n < 0)
  {
    return 0;
  } else 
  {
	return v[n - 1] + soma(v, n-- - 1);   
  }
}
\end{lstlisting}

\end{Exercise*}

\end{document}